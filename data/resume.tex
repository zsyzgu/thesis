% !TeX root = ../thuthesis-example.tex

\begin{resume}

  \section*{个人简历}

1995年5月12日出生于广东省中山市

2013年9月考入清华大学计算机系计算机专业,2017年7月本科毕业并获得学士学位。

2017年9月保研进入清华大学计算机系攻读人机交互专业博士至今。

 \section*{在学期间完成的相关学术成果}

 \subsection*{学术论文}

第一作者发表论文

\begin{achievements}
 	\item \textbf{Gu Y}, Yu C, Chen X, et al. Typeboard: Identifying unintentional touch on pressure-sensitive touchscreen keyboards[C]//The 34th Annual ACM Symposium on User Interface Software and Technology. 2021: 568-581. (TH-CPL Rank A)
 	\item \textbf{Gu Y}, Yu C, Li Z, et al. Qwertyring: Text entry on physical surfaces using a ring[J]. Proceedings of the ACM on Interactive, Mobile, Wearable and Ubiquitous Technologies, 2020, 4(4): 1-29. (TH-CPL Rank A)
 	\item \textbf{Gu Y}, Yu C, Li Z, et al. Accurate and low-latency sensing of touch contact on any surface with finger-worn imu sensor[C]//Proceedings of the 32nd Annual ACM Symposium on User Interface Software and Technology. 2019: 1059-1070. (TH-CPL Rank A)

\end{achievements}

共同一作发表论文

\begin{achievements}
	\item Liu G, \textbf{Gu Y}, Yin Y, et al. Keep the phone in your pocket: Enabling smartphone operation with an imu ring for visually impaired people[J]. Proceedings of the ACM on Interactive, Mobile, Wearable and Ubiquitous Technologies, 2020, 4(2): 1-23. (TH-CPL Rank A)
\end{achievements}

学生一作发表论文

\begin{achievements}
	\item Yu C, \textbf{Gu Y}, Yang Z, et al. Tap, dwell or gesture? exploring head-based text entry techniques for hmds[C]//Proceedings of the 2017 CHI Conference on Human Factors in Computing Systems. 2017: 4479-4488. (TH-CPL Rank A)
\end{achievements}

非第一作者发表论文

\begin{achievements}
	\item Yu C, Wei X, Vachher S, Qin Y, Liang C, Weng Y, \textbf{Gu Y}, et al. Handsee: enabling full hand interaction on smartphone with front camera-based stereo vision[C]//Proceedings of the 2019 CHI Conference on Human Factors in Computing Systems. 2019: 1-13. (TH-CPL Rank A)
\end{achievements}

\subsection*{专利}

\begin{achievements}
\item 史元春, 喻纯, \textbf{古裔正}. 2021.07. 一种抬起手势的识别方法、系统、电子设备及存储介质. CN111580664A. (中国专利申请号,已授权)
\item 史元春, 喻纯, 秦岳, \textbf{古裔正}, 韦笑颖. 2021.07. 智能电子设备上基于单摄像头的双目视觉与物体识别技术. CN109993059B. (中国专利申请号,已授权)
\item 喻纯, \textbf{古裔正}, 杨志灿, 阎裕康, 史元春. 2018.08. 手型跟踪装置. CN207752443U. (实用新型,已授权)
\item 史元春, 喻纯, \textbf{古裔正}, 周诚驰, 张磊. 2022.01. 一种二维码扫描方法、装置及电子设备. CN113935348A.(中国专利公开号,已公开)
\item 史元春, 喻纯, \textbf{古裔正}, 周诚驰, 张磊. 2022.01. 一种控制智能家电的方法、装置及电子设备. CN113934150A.(中国专利公开号,已公开)
\item 史元春, 喻纯, \textbf{古裔正}. 2021.11. 触摸屏防误触的方法及装置、电子设备及存储介质. CN113608634A. (中国专利公开号,已公开)
\item 史元春, 喻纯, \textbf{古裔正}. 2020.08. 一种信息输入方法、系统、电子设备及存储介质. CN111580663A. (中国专利公开号,已公开)
\item 史元春, 喻纯, 刘冠宏, \textbf{古裔正}.2020.08. 一种设备控制方法、电子设备、设备控制系统及存储介质. CN111580666A.(中国专利公开号,已公开)
\end{achievements}

\end{resume}
