% !TeX root = ../thuthesis-example.tex

% 中英文摘要和关键字

\begin{abstract}

% [V1] 触摸输入是自然人机交互的重要组成部分,在过去和当下,人通过触屏设备向计算机输入信息。随着AR头盔、智能手表等可穿戴设备的发展,交互向着普适计算的方向发展,在未来,人能在任何时间地点、以任何形式进行信息的获取和处理。在此背景下,本文提出\textbf{普适触摸输入}的概念:\textit{人能在任何表面,如桌面、墙面和手掌上通过触摸向计算机输入信息;触摸输入感知摆脱对触屏的依赖,转而利用可穿戴设备等普适计算单元识别}。本文研究普适触摸输入的感知技术,有两大难点:(1)难点一是判断人是否触摸了表面。人对触摸交互的敏感程度高,在判别准确率、延迟和定位精度这三方面给触摸感知提出了高要求。在普适触摸模态下,不可能在任意表面上加装传感器,而是利用普适计算单元识别,感知难度大大提升。(2)难点二是判断触摸是否表达交互意图。当人能在任意表面触摸输入,表面就同时具有物理属性和数字属性,人触摸表面具有二义性,既可能表达数字交互意图,也可能是利用表面的物理属性。因此,系统需要判断触摸是否表达交互意图,防止误触。本文基于普适触摸输入的行为模型,提出了多信道融合的触摸感知技术,准确、快速感知普适触摸,同时利用连续触摸的时空相关性表征人的触摸行为,有效辨识触摸是否表达交互意图。具体研究贡献包括:(1)提出了触摸的多信道行为模型:本文研究了触摸的行为模型,从视觉、听觉、震动等多个信道总结了触摸瞬间的客观物理现象规律。(2)提出了多信道融合的触摸感知技术:针对触摸感知在判别、延迟和定位这三方面的性能需求,本文结合各信道上突出的信号特征,实现了高性能的普适触摸输入感知技术,判别准确率大于98.5\%,延迟小于10毫秒,定位误差小于1厘米。(3)提出了多信道融合的触摸意图推理技术:在普适触摸输入模态下,系统需要鉴别触摸是否表达数字意图,防止误触。本文提出利用连续触摸的时空相关性表征触摸行为的方法,实现了准确的普适触摸输入意图推理技术。在最复杂的触摸输入任务——文本输入实验的验证下,该技术的防误触能力达到98.9\%。

%探索触摸前后多维度行为数据的时空相关性特征,本文针对多种触摸交互任务(目标选择、文本输入)提高交互效率,本文改进了触摸的感知和意图推理技术,从而改善用户体验。触摸的多信道融合模型为触摸的感知和交互技术提供了有效的计算理论基础,在此融合的模型的支持下,本文从三个方向改进了触摸感知和交互技术,具体贡献如下:

%如何建模和利用触摸的多信道特征,是触摸交互的重要研究问题——多信道融合技术将在广度和深度上改进触摸交互技术:(1)广度上,基于多信道融合的穿戴式传感技术将支持普通表面上的触摸感知,赋能桌面、墙面上的触摸交互,使触摸交互摆脱对触屏的依赖;(2)深度上,基于多信道融合的意图推理技术将准确区分有意触摸输入和无意触碰。本文提出触摸的多信道融合模型,探索触摸在视觉、声音、震动和压力等维度上的时空相关性特征,针对多种触摸交互任务(目标选择、文本输入)提高交互效率,改进触摸的感知和意图推理技术,从而改善用户体验。具体研究贡献包括:

%(1)提出了触摸的多信道融合模型:针对相关工作在触摸的视觉、声音、震动、压力等信道上存在空缺的研究现状,本文提出了触摸的多信道融合模型,为触摸交互技术的改进提供了计算理论基础。该模型是在广度和深度上提升触摸交互能力的利器,以下两条贡献均为其应用实例。

【题目】基于最优控制理论的触摸运动模型

触摸交互是重要的人机交互方式,是人控制手指触摸交互表面,通过点击、长按、滑动等手势向计算机输入信息的方式。尽管触摸交互的主要载体触摸屏已问世数十年,触摸交互仍存在三点不足:(1)普适性:触摸交互被限制于有源表面上,不满足普适计算中人随时随地与计算机交互的需求;(2)响应性:人对触摸交互响应性的感官需求极高,能察觉到低至10毫秒的延迟,而常用触摸屏的延迟为50毫秒,不能提供最佳用户体验;(3)意图性:人机交互朝着自然动作输入的方向发展,而自然触摸交互中人的有意触摸和无意触碰混杂,误触更频繁。为克服上述不足,设计无源、低延迟、精准的触摸交互技术,本文提出了基于最优控制理论的触摸运动模型,揭示触摸前后极短时间内手指的运动规律。该模型为基于运动传感(包括位移、速度和加速度)的触摸交互技术提供了计算理论基础。基于模型,本文针对多种触摸交互任务(目标选择、文本输入),优化触摸感知和防误触技术,提升交互效率和用户体验。具体贡献如下:

(1)提出了基于最佳控制理论的触摸运动模型,包含数学模型(描述触摸运动的数学方程)和计算模型(利用位移、速度、加速度传感信号拟合触摸运动方程的计算方法)。本文详细描述了模型的表达、提出过程和推导过程,通过实验验证了模型的有效性,最后讨论了模型在普适性、响应性和意图性三方面对触摸交互技术的指导意义。

(2)提出了基于运动传感指环的低延迟触摸感知技术:基于指环的触摸感知技术使用户能在无源表面(如桌面、墙面)上触摸输入[xx],具有较强的普适性。先前工作利用阈值方法感知触摸[xx],延迟高达200毫秒,准确率仅为85\%[xx]。本文基于触摸运动模型,提出了低延迟的基于指环的触摸感知技术,延迟低于10毫秒,准确率超过99\%。低延迟是技术的关键,由于用户无法在触摸交互中察觉到低于10毫秒的延迟,该技术理论上提供了最佳用户体验。

(3)提出了面向连续触摸输入的防误触技术:文本输入是最复杂的连续触摸输入任务,针对触摸屏十指文本输入中用户频繁误触的问题,本文提出基于触摸运动模型的防误触技术。本文通过文本输入任务评测防误触技术的性能,防误触技术的准确率达到99\%,且允许用户在打字时将非交互手指休息在触摸屏上(而不会引发误触),将打字速率提升20\%。

% (4)触摸运动模型为触摸交互技术提供了计算理论基础,对触摸交互技术具有指导性意义。例如,众多研究者利用视觉方法识别触摸,却受困于视线遮挡问题[xx]。我们的模型指出,触摸瞬间手指发生毫秒级的瞬停现象,因此触摸感知不一定要观察到接触本身,而可以检测手指是否瞬停,从而绕过遮挡问题。

  % 关键词用“英文逗号”分隔,输出时会自动处理为正确的分隔符
  \thusetup{
    keywords = {触摸交互, 感知技术, 意图识别},
  }
\end{abstract}

\begin{abstract*}
  An abstract of a dissertation is a summary and extraction of research work and contributions.
  Included in an abstract should be description of research topic and research objective, brief introduction to methodology and research process, and summary of conclusion and contributions of the research.
  An abstract should be characterized by independence and clarity and carry identical information with the dissertation.
  It should be such that the general idea and major contributions of the dissertation are conveyed without reading the dissertation.

  An abstract should be concise and to the point.
  It is a misunderstanding to make an abstract an outline of the dissertation and words “the first chapter”, “the second chapter” and the like should be avoided in the abstract.

  Keywords are terms used in a dissertation for indexing, reflecting core information of the dissertation.
  An abstract may contain a maximum of 5 keywords, with semi-colons used in between to separate one another.

  % Use comma as separator when inputting
  \thusetup{
    keywords* = {keyword 1, keyword 2, keyword 3, keyword 4, keyword 5},
  }
\end{abstract*}
